\documentclass{ctexart}

\usepackage{graphicx}
\usepackage{amsmath}

\title{作业一: L'Hôpital's rule的叙述与证明}


\author{赖泽宇 \\ 统计学3190102225}

\begin{document}

\maketitle


洛必达法则是一个用于简化不确定形式极限的工具,它可以将不确定形式的求极限问题
简化为另一种易求的形式。
\section{问题描述}
如果满足以下四个条件
\begin{itemize}
    \item 
    \begin{equation}\label{eqa::1}
        \lim\limits_{x\to c}f(x)=\lim\limits_{x\to c}g(x) = 0 
    \end{equation}
        \\ or \\
    \begin{equation}\label{eqa::2} 
        \lim\limits_{x\to c}f(x)=\lim\limits_{x\to c}g(x) = \pm \infty 
    \end{equation}
    \item $f(x)$ 和 $g(x)$ 在一个包含该点$c$的开区间$\mathcal{I}$上可微
    \item $\forall x \neq c, \ x \in \mathcal{I}, \ g^{'}(x) \neq 0$
    \item $\lim\limits_{x\to c}\frac{f^{'}(x)}{g^{'}(x)}$ 存在
\end{itemize}
则有 $\lim\limits_{x \to c}\frac{f(x)}{g(x)} = \lim\limits_{x\to c}\frac{f^{'}(x)}{g^{'}(x)}$

\section{证明}
下仅给出情况(\ref{eqa::1})的证明。\\
设两函数$f(x)$及$g(x)$在$c$点附近连续可导,$f(x)$及$g(x)$都在$c$点连续,
且值皆为$0$,即:
$$
f(c) = g(c) = 0; \ \lim\limits_{x\to c}f(x) = 0; \ \lim\limits_{x\to c}g(x) = 0
$$
另一方面,两函数的导数比值在$c$点存在,记为:
\begin{equation}\label{result::1}
\lim\limits_{x\to c}\frac{f^{'}(x)}{g^{'}(x)} = L
\end{equation}
由极限的定义,对$\forall \epsilon > 0, \exists \eta > 0 $,
s.t.$\forall x\ neq c, \ c - \eta \leq x \leq c + \eta$:
$$
L - \epsilon \leq \frac{f^{'}(x)}{g^{'}(x)} \leq L + \epsilon 
$$
根据柯西中值定理,$\forall x\neq c, c - \eta \leq x c + \eta$, $\exists \xi $介于$a$和$x$,s.t.
\begin{equation}\label{eqa::3}
    \frac{f(x)}{g(x)} = \frac{f(x) - f(c)}{g(x) - g(a)} = \frac{f^{'}(\xi)}{g^{'}(\xi)}
\end{equation}
于是, $L-\epsilon \leq \frac{f(x)}{g(x)} \leq L + \epsilon$\\
因此, $\lim\limits_{x\to c}\frac{f(x)}{g(x)}$与(\ref{result::1})相等,命题得证。

\end{document}

